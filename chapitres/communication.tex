
\section{Les presentations: Beamer, Powerpoint ou Keynote}

\mytodo{Jo}

\section{Cr\'eation graphique}
Pour les figures il est bon de connaître un
logiciel au moins pour la retouche d'image ou pour la création de schéma.
On pourra utiliser des outils ``open source'' tel  \href{http://inkscape.org/}{Inkscape}
 dont le format vectoriel d'export est .svg, un standard, et qui supporte l'insertion de symboles \LaTeX. 
 On peut consulter: \url{http://famelis.wordpress.com/2011/04/12/latex-and-inkscape/}
pour plus d'information.

En guise d'exemple la figure \ref{fig:smv} est produite par la commande:
\begin{lstlisting}
\begin{figure}[htb] 
\centering \def\svgwidth{200pt} 
\input{images/separateur_jo.pdf_tex} 
\caption{Marge et hyperplan s\'parateur (cas lin\'aire)} 
\label{fig:smv}
\end{figure}
\end{lstlisting}
Dans le cas de la création de ce document, on dispose dans un sous-répertoire local ``/image'',
d'un fichier \lstinline+separateur_jo.pdf_tex+ et d'un fichier \lstinline+separateur_jo.tex+
tous deux crées par \href{http://inkscape.org/}{Inkscape}.
Attention tout de m\^eme dans l'utilisation de sous dossiers,
il faut éventuellement penser \`a éditer le fichier
\lstinline+separateur_jo.tex+
et a changer la commande \lstinline+separateur_jo.pdf_tex+
par \lstinline+images/separateur_jo.pdf_tex+


\begin{figure}[htb] 
\centering \def\svgwidth{400pt} 
\input{images/separateur_jo.pdf_tex} 
\caption{Marge et hyperplan séparateur (cas linéaire)} 
\label{fig:smv}
\end{figure}


\section{Création de sites web: visibilité en ligne}
Il est important en recherche non seulement de faire des travaux de bonne qualité
mais encore de rendre ces travaux disponibles facilement sur Internet. Cela peut passer par la création
d'un site personnel sur le site de votre employeur (ou bien par la création d'un site personnel),
la création d'un compte \href{http://scholar.google.com}{Scholar Google}, ou encore la soumission de papier
sur \href{http://hal.archives-ouvertes.fr/}{Hal} ou \href{http://arxiv.org/}{arXiv}.
Il est également important de fournir en ligne les codes numériques qui permettent de recréer les 
expériences commentées dans un article scientifique. Cette motivation se nomme 
la recherche reproductible (\lcf par exemple le site \href{http://www.ipol.im/}{IPOL} dans le domaine
du traitement des images)  et vise à fournir comme dans les sciences expérimentales des expériences 
que l'on peut reproduire à souhait. C'est la seule manière de garantir que d'autres experimentateurs
peuvent reproduire nos propres résultats. Hélas, cette phase cruciale n'est pas toujours complètement 
prise en compte par les chercheurs de peur que quelqu'un trouve un \textit{bug} dans leur code.
De plus au cours du processus de relecture des travaux de recherches (\textit{review}) cet
aspect est parfois complètement délaissé par l'équipe éditoriale.

\begin{itemize}
 \item  html:
 \item  css:
 \item  php:
\end{itemize}


\mytodo{Jo, Benoit.}