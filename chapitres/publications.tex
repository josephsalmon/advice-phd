%%%%%%%%%%%%%%%%%%%%%%%%%%%%%%%%%%%%%%%%%%%%%%%%%%%%%%%%%%%%%%%%%%%%%%%%%%%%%%%%%%%%%%%%%%%%%%%%%%%%%%%%%
\section{Journaux importants}
%%%%%%%%%%%%%%%%%%%%%%%%%%%%%%%%%%%%%%%%%%%%%%%%%%%%%%%%%%%%%%%%%%%%%%%%%%%%%%%%%%%%%%%%%%%%%%%%%%%%%%%%%


\mytodo{expliquer le fonctionnement ``à l'aveugle''}

Toutes les journaux distribuent à des relecteurs (\textbf{reviwer})
les articles qu'ils reçoivent, ces derniers n'étant pas connus des auteurs pour garantir une
certaine tranquillité dans les discussions. Selon les cas de figures les papiers peuvent passer
par plusieurs tours de relecture (en général de un à trois) qui permettent d'améliorer 
le contenu autant que possible.

\subsection{Statistiques}
\begin{itemize}
 \item Annals of Statistics
 \item Bernouilli
 \item Probability  Theory and Related Fields (PTRF)
 \item Statistical Science
 \item Electronic Journal of Statistics (EJS)
\end{itemize}

\subsection{Apprentissage statistique}
\begin{itemize}
 \item Journal of Machine Learning Research
 \item Machine Learning
\end{itemize}

\subsection{Vision}
\begin{itemize}
\item International Journal Computer Vision
\end{itemize}

\subsection{Traitement des images et du signal}
\begin{itemize}
 \item SIAM Imaging Science
 \item Pattern Analysis and Machine Intelligence 
 \item Trans. Image Processing
 \item Journal of Mathematical Imaging and Vision
 \item Signal Processing
\end{itemize}




%%%%%%%%%%%%%%%%%%%%%%%%%%%%%%%%%%%%%%%%%%%%%%%%%%%%%%%%%%%%%%%%%%%%%%%%%%%%%%%%%%%%%%%%%%%%%%%%%%%%%%%%%
\section{Conférences importantes}
%%%%%%%%%%%%%%%%%%%%%%%%%%%%%%%%%%%%%%%%%%%%%%%%%%%%%%%%%%%%%%%%%%%%%%%%%%%%%%%%%%%%%%%%%%%%%%%%%%%%%%%%%
Il y a principalement deux types de revues: celle qui pratique le double aveugle (\textit{double blind}),
souvent les plus prestigieuses, et les autres. 
Toutes les conférences distribuent à des relecteurs 
les articles qu'elles reçoivent, ces derniers  n'étant pas connus des auteurs. 
Dans le système du double aveugle, les papiers sont envoyés aux
relecteurs et le nom des auteurs est alors inconnu de ces relecteus (du moins en théorie).

\mytodo{expliquer le fonctionnement ``double aveugle'', double blind}


%%%%%%%%%%%%%%%%%%%%%%%%%%%%%%%%%%%%%%%%%%%%%%%%%%%%%%%%%%%%%%%%%%%%%%%%%%%%%%%%%%%%%%%%%%%%%%%%%%%%%%%%%
\subsection{Statistiques}

\begin{itemize}
 \item AISTATS
\end{itemize}

\subsection{Apprentissage statisitiques}
\begin{itemize}
 \item ICML, NIPS, COLT, ALT
\end{itemize}
\subsection{Vision}

\begin{itemize}
 \item ICCV, CVPR 
\end{itemize}


%%%%%%%%%%%%%%%%%%%%%%%%%%%%%%%%%%%%%%%%%%%%%%%%%%%%%%%%%%%%%%%%%%%%%%%%%%%%%%%%%%%%%%%%%%%%%%%%%%%%%%%%%
\subsection{Traitement des images et du signal}

\begin{itemize}
 \item ICIP, SSP, ICASSP, SIAM Imaging
\end{itemize}


%%%%%%%%%%%%%%%%%%%%%%%%%%%%%%%%%%%%%%%%%%%%%%%%%%%%%%%%%%%%%%%%%%%%%%%%%%%%%%%%%%%%%%%%%%%%%%%%%%%%%%%%%
\section{\'Ecoles d'\'et\'e et rencontres de jeunes chercheurs}
%%%%%%%%%%%%%%%%%%%%%%%%%%%%%%%%%%%%%%%%%%%%%%%%%%%%%%%%%%%%%%%%%%%%%%%%%%%%%%%%%%%%%%%%%%%%%%%%%%%%%%%%%


%%%%%%%%%%%%%%%%%%%%%%%%%%%%%%%%%%%%%%%%%%%%%%%%%%%%%%%%%%%%%%%%%%%%%%%%%%%%%%%%%%%%%%%%%%%%%%%%%%%%%%%%%
\subsection{Statistiques}
Ce genre de rencontres permet aux jeunes chercheurs de rencontrer ses collègues d'une m\^me génération
dans une atmosphère plus détendue et plus propice aux collaborations et échanges informels que les
grandes conférences. 
\begin{itemize}

\item \textbf{Les Rencontres des Jeunes Statisticiens}: cet \'ev\`enement a lieu \`a Aussois (Savoie) 
tous les deux ans (ann\'ees impaires), 
g\'en\'eralement fin août. \url{http://www.sfds.asso.fr/111-Les_Rencontres_des_Jeunes_Statisticiens}

\item Le colloque  \textbf{Jeunes Probabilistes et Statisticiens} qui a lieu tous les deux ans \'egalement 
(ann\'ees paires) et permet de rencontrer un public plus large ouvert aussi aux probabilistes. \url{http://jps.math.cnrs.fr/2014/}

\item L'école d'été de St-Flour \url{http://math.univ-bpclermont.fr/stflour/}

\end{itemize}


%%%%%%%%%%%%%%%%%%%%%%%%%%%%%%%%%%%%%%%%%%%%%%%%%%%%%%%%%%%%%%%%%%%%%%%%%%%%%%%%%%%%%%%%%%%%%%%%%%%%%%%%%
\subsection{Traitement des images et du signal}



\begin{itemize}
\item \'Ecole d'été de Peyresq, rendez-vous annuel:
\url{http://www.gretsi.fr/ecole-d-ete-de-peyresq.php}
\end{itemize}


%%%%%%%%%%%%%%%%%%%%%%%%%%%%%%%%%%%%%%%%%%%%%%%%%%%%%%%%%%%%%%%%%%%%%%%%%%%%%%%%%%%%%%%%%%%%%%%%%%%%%%%%%
\section{Séminaires (parisiens)}
%%%%%%%%%%%%%%%%%%%%%%%%%%%%%%%%%%%%%%%%%%%%%%%%%%%%%%%%%%%%%%%%%%%%%%%%%%%%%%%%%%%%%%%%%%%%%%%%%%%%%%%%%

%%%%%%%%%%%%%%%%%%%%%%%%%%%%%%%%%%%%%%%%%%%%%%%%%%%%%%%%%%%%%%%%%%%%%%%%%%%%%%%%%%%%%%%%%%%%%%%%%%%%%%%%%
\subsection{Statistiques}

\begin{itemize}
 \item Le lundi après midi, une fois par mois:
 \href{https://sites.google.com/site/semstats/}{Séminaire Parisien de Statistique}
 \item Le lundi et jeudi, deux fois par mois:  
\href{https://sites.google.com/site/smileinparis/home}{SMILE (Statistical Machine Learning in Paris)}
 \item Le lundi: 
\href{http://certis.enpc.fr/~dalalyan/seminar.html}{Séminaire statistique de l'ENSAE-CREST}
 \item Le lundi une fois par mois environ \`a l'universit\'e Paris Diderot-Paris 7:  
\href{http://dominiquepicard.blogspot.fr/p/seminaire-point-de-vue.html}{Point de vue}
\end{itemize}



%%%%%%%%%%%%%%%%%%%%%%%%%%%%%%%%%%%%%%%%%%%%%%%%%%%%%%%%%%%%%%%%%%%%%%%%%%%%%%%%%%%%%%%%%%%%%%%%%%%%%%%%%
\subsection{Images}
\begin{itemize}
 
 \item Le vendredi matin \`a Paris 5, une fois par mois:
 \href{http://w3.mi.parisdescartes.fr/map5/-Groupe-de-travail-Modelisation-}
{Groupe de travail Modélisation numérique et Images
}
\end{itemize}
