Il est rare de travailler entièrement seul dans le monde académique et écrire un document ou un code à plusieurs 
pose quelques problèmes de synchronisation et de sauvegarde. C'est pour cela qu'on utilise des logiciels de gestion de version.
Ces logiciels sont utiles également pour des projets en solitaire, car ils permettent de conserver tous les modifications
que vous apportez à un document, ce qui élimine les risques d'erreurs et de perte de votre travail. 


\section{SVN}
Ce syst\`eme est plutôt ancien, mais est souvent  disponible dans votre laboratoire, université ou école.
\mytodo{Jo , Charles}

\section{Git}
\mytodo{Alex, Yann}

Git est un logiciel de gestion de version décentralisé proposé par Linus Torvald pour
gérer le développement du noyau Linux. C'est un système extrêmement en vogue et la plupart des
 grands projets de programmation sont passés à Git. Le logiciel est parfois complexe et sans doute
 inutilement puissant pour rédiger un article à 3 ou 4. Mais, du fait de sa popularité, il y a de nombreuses
 ressources pour apprendre Git sur internet et il est très facile de monter un projet git sur un site gratuit.
 
 
 Des ressources pour apprendre Git :
 \begin{itemize}
  \item Un livre gratuit pour apprendre git \url{http://git-scm.com/book}
  \item Plusieurs sites d'apprentissage interactif par exemple  \url{http://try.github.io}
  %\item
 \end{itemize}

 
 Le mieux est de créer votre dépôt git sur un serveur de votre laboratoire si c'est possible.
 Sinon, utilisez un site comme \href{http://gitlab.org/}{Gitlab} qui permet de créer gratuitement des dépôts.