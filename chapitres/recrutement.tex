

Plusieurs chercheurs connus ont partagé leur vision de la recherche
pour éclairer les débutants (et même les autres).
\begin{itemize}
 \item Les conseils de Terence Tao sur la \href{http://terrytao.wordpress.com/career-advice/}{carrière de mathématicien} et sur \href{http://terrytao.wordpress.com/advice-on-writing-papers/}{l'écriture d'articles}.
\end{itemize}


Un point important qu'il est utile d'aborder est : comment trouver un emploi dans la recherche?
Il existe un certains nombre de ressource pour cela
\href{http://guilde.jeunes-chercheurs.org/}{la guilde du doctorant}: France




Deux site généraux pour tous types d'emplois:

\href{http://www.phds.org/}{Phds.org}: international

\href{https://www.mathjobs.org/jobs}{MathJobs}: international


\section{Post-doc}
En parler avec son directeur, d'anciens doctorants, des coll\`egues, AVANT de soutenir sa th\`ese:
cela \'evite de nombreux désagréments.
\mytodo{Jo,Charles}

\section{Chercheurs: concours CNRS, Inria}
\mytodo{Charles: CNRS, INRIA}

\section{Maître de conférences (France)}
\mytodo{Jo,Yann,}

Ne pas oublier les QUALIFS!!!!

\href{http://postes.smai.emath.fr/}{Opération postes}: France

\href{https://www.galaxie.enseignementsup-recherche.gouv.fr/ensup/candidats.html}{Galaxie}: France

\section{Assistant Professor (France)}

\href{http://notable.math.ucdavis.edu/wiki/Mathematics_Jobs_Wiki}{Mathematics Jobs Wiki} principalement
pour les \'Etats-Unis

\mytodo{Jo,Yann,Alex}