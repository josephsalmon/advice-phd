

La gestion de la bibliographie est une étape importante à automatiser dans la perspective 
de la rédaction d'une thèse qui comprendra vraisemblablement plusieurs centaines de références.
Il est recommandé d'utiliser bibtex pour cela. De plus, il est bon de veiller à une harmonisation
des noms des références une bonne fois pour toute. Un conseil est de plutôt garder un fichier 
qui contient toutes les entrées que l'on fait grossir au fur et \`a mesure de ces lectures.
De plus il est bon de veiller à une connexion entre le fichier contenant les noms des références
et le dossier dans lequel on garde ces références. Sous Linux un bon candidat pour gérer
cette interconnexion est \href{http://jabref.sourceforge.net/}{Jabref}.


Pour les références bibliographiques,  il faut créer un autre fichier d'extension .bib, qui est 
l'extension des fichiers ``\textbf{bibtex}''. On le met également dans le dossier qui contient le 
fichier .tex que l'on rédige. Le fichier de la bibliographie associée à ce document est disponible
 \href{http://josephsalmon.eu/enseignement/M1/refs.bib}{là}.  On présente les articles, 
livres et autres sites webs de la façon suivante : (on peut trouver également ce type d'information
 tout rempli sur internet, comme sur le site \href{http://ams.u-strasbg.fr/mathscinet/}{MathSciNet}
 par exemple.)\medskip
\begin{lstlisting}
@inproceedings{Tsybakov03, 
  author    = {A. B. Tsybakov},
  title     = {Optimal Rates of Aggregation}, 
  booktitle = {COLT}, 
  year      = {2003}, 
  pages     = {303-313}, 
}
\end{lstlisting}

On doit alors compiler la bibliographie avec la commande bibtex. 
Selon le logiciel, ou la mani\`ere de faire,   (par exemple dans Kile) il faut aller dans l'onglet 
``\textbf{build}'' ou ``\textbf{construire}'', puis cliquer sur ``\textbf{compile}''. Le type de 
compilation est alors Bibtex.\medskip

La citation apparaît à l'endroit voulu grâce à la commande  \lstinline+\ref{nom_citation}+. 
Tout à la fin du fichier .pdf, on trouve alors  les références voulues avec tous les détails que l'on
 a précisés 
sur le document cité. 
Pour cela il faut  écrire juste avant \lstinline+\end{document}+ les commandes 
\lstinline+\bibliography{nom du fichier de la bibliographie}+
\lstinline+\bibliographystyle{alpha}+ pour que les citations apparaissent avec des lettres et l'année
 et non juste des numéros. Par exemple on peut parler de l'article \cite{Tsybakov03} ou bien du livre 
\cite{Catoni04}. Les articles non cités n'apparaissent pas. \medskip

On remarquera qu'à la fin de la bibliographie, les pages où une référence est citée apparaissent sous
 forme de liens hypertextes. C'est l'utilisation du package \lstinline+pagebackref+ qui rend cela possible.
Pour plus d'informations sur Bibtex en français : 
\href{http://www.irit.fr/~Alain.Crouzil/jaffre/LOGICIELS/LATEX_BIBTEX/bibtex1.html}{Le site de Gaël Jaffré},
\href{http://fr.wikipedia.org/wiki/BibTeX}{Wikipedia},  
et en anglais :  \href{http://www.bibtex.org/}{www.bibtex.org} 
