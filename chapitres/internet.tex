
La plupart des ressources utiles sont maintenant disponibles sur le réseau,
même si la fréquentation des bibliothèques est aussi une bonne pratiques.


\section{Sites généraux}
\begin{itemize}
 \item \href{www.google.com}{Google} : 
no comment à part que paramétrer son compte
peut aider (langue par défaut, recherche détaillée avec les signes `` '' ou -).

\item \href{http://wikipedia.org/}{Wikipedia} : no comment, 
bons pour tous les niveaux, spécialement en guise d'introduction. Pour les maths,
la version française est souvent plus detaillée.

\end{itemize}




\subsection{Dictionnaire et traducteurs}

\begin{itemize}
\item \href{http://www.linguee.com/}{Linguee} : dictionnaire franco-anglais très bon, qui propose 
surtout des phrases où les mots sont contextualisés. Très précis.
\item \href{http://www.lexilogos.com/}{Lexilogos} : site référençant des dictionnaires et autres
aides (grammaire, conjugaison) pour un très grand nombre de langues. 
\item \href{https://translate.google.fr/}{Google translate} : site s'améliorant avec le temps, 
il a l'avantage d'aider à la prononciation.
\end{itemize}


\subsection{Biblioth\`eque en ligne}

\begin{itemize}

\item \href{http://catalogue-bibliotheques.upmc.fr/#focus}{Biblioth\'eque de l'UPMC} : Ressources en 
ligne et aussi disponibles sur le site de Jussieu et de Sophie-Germain.

\item des sites russes ou chinois que l'on ne saurait nommer.

\end{itemize}


\section{Sites scientifiques spécialisés}



\begin{itemize}
\item \href{http://scholar.google.com}{Scholar Google} :
Ouvrir un compte peut être une bonne idée pour augmenter sa visibilité en recherche,
et retrouver facilement ses co-auteurs sur Internet.
Il est aussi bon de paramétrer \href{http://scholar.google.com}{Scholar Google} pour
obtenir les fichiers  .bib (cf. les parties sur Latex en chapitre ... \mytodo{link})

\item \href{http://arxiv.org/}{arXiv} :
À suivre régulièrement (par exemple en consultant les flux RSS avec 
\href{https://www.newsblur.com/}{NewsBlur}). 
C'est le lieu international
où les gens déposent leurs pré-publications. C'est donc l'endroit conseillé pour faire
de même.

Potentiellement il peut être recommandé de suivre avec une certaine fréquence ce qui est soumis
chaque jour, voir profiter d'un groupe de lecture régulier pour discuter des papiers récents les plus
marquants.

\item \href{http://hal.archives-ouvertes.fr/}{Hal} : c'est l'équivalent français d'arXiv.
Les thèses françaises y sont toutes référencées.


\item \href{http://www.ams.org/mathscinet/}{Mathscinet} : référence les travaux publiés dans
des journaux de mathématiques. Utiles aussi pour trouver des fichies .bib, ainsi que les sources
des papiers. Il manque toute la partie de littérature du cote des ``computer sciences'' (sciences
computationnelles ou informatique) ainsi que les revues de type signal.
\end{itemize}





\subsection*{R\'esum\'e des sites utiles mentionés dans ce chapitre}

\begin{itemize}
\item[\color{orange_js}{$\startri$}]
\url{http://arxiv.org/}
 \item[\color{orange_js}{$\startri$}]
\url{http://scholar.google.com}
 \item[\color{orange_js}{$\startri$}]
\url{http://scholar.google.com} \item[\color{orange_js}{$\stardble$}]
\url{http://hal.archives-ouvertes.fr/}
  \item[\color{orange_js}{$\stardble$}]
\url{http://jrjohansson.github.io/}
\end{itemize}
